\documentclass[11pt]{article}
\usepackage{xcolor}
\usepackage[english]{babel}
\usepackage{fancyhdr}
%\usepackage{tabularx}
%\usepackage{tablestyles}
\usepackage[hidelinks=true]{hyperref}

% Reverse Numbering
\usepackage{etaremune}

% Set 1-inch margins.
\usepackage[margin=1in]{geometry}

% Headers and Footer
\pagestyle{fancy}
\fancyhf{}
\rhead{\textrm{\textbf{William E.\ Fondrie}}}
\lhead{\textrm{\textbf{Curriculum Vitae}}}
\rfoot{\thepage}
\renewcommand{\headrulewidth}{0pt}

\fancypagestyle{plain}{
  \fancyhf{}
  \rfoot{\thepage}
  \renewcommand{\headrulewidth}{0pt}
}

% Use Roboto font
\usepackage[rm]{roboto}
\usepackage[T1]{fontenc}
\renewcommand{\familydefault}{\sfdefault}

% Skip a line between paragraphs.
\setlength{\parskip}{1.5ex}

% Turn off indenting.
\setlength{\parindent}{0in}

% Define the section command.
\newcommand{\mysection}[1]{\vspace{1ex {\bf \large \textrm{#1}} \quad
    \hrulefill}}
\newcommand{\mysubsection}[1]{\vspace{1ex {\bf \textrm{#1}}}}

% Set the numbering style.
\renewcommand{\labelenumi}{(\arabic{enumi})}

% Make the \footnote command print symbols (*, etc).
\renewcommand{\thefootnote}{\fnsymbol{footnote}}

% Smarter version of href.
\newcommand{\myref}[1]{\href{#1}{\url{#1}}}

% Table sizes
\newlength{\leftcol}
\setlength{\leftcol}{0.6in}
\newlength{\rightcol}
\setlength{\rightcol}{5.25in minus \leftcol}
\newcommand{\tdim}{p{\leftcol}p{\rightcol}}
\renewcommand{\arraystretch}{1.25} % make rows taller

% No hyphenation
\pretolerance=10000
\raggedright

% colors
\definecolor{gray}{HTML}{404040}
\color{gray}

%%%%%%%%%%%%%%%%%%%%%%%%%%%%%%%%%%%%%%%%%%%%%%%%%%%%%%%%%%%%%%%%%%%%%%%%%%%%%%%%
\begin{document}
\thispagestyle{plain}
\begin{center}
  {\bf \Large \textrm{William E Fondrie, PhD}}
  
  Department of Genome Sciences\\
  University of Washington\\
  wfondrie@uw.edu \\
  % {phone #}\\
  \href{https://willfondrie.com}{willfondrie.com}\\
  Updated \today.
\end{center}


\mysection{Education}
\begin{tabular}{\tdim}
  % 9/2013--06/2018
  2013--18 & {\bf PhD in Molecular Medicine}
             \newline University of Maryland, Baltimore
             \newline Advisors: David Goodlett, PhD and Dudley Strickland, PhD\\
  % 9/2009--05/2013
  2009--13 & {\bf BS in Chemistry}
             \newline University of North Carolina at Chapel Hill\\
\end{tabular}

\mysection{Employment and Professional Appointments}

\begin{tabular}{\tdim}
  2019--   & {\bf University of Washington Data Science Postdoctoral Fellow}\\
  2018--   & {\bf Postdoctoral Fellow}, University of Washington
             \newline Advisor: William Noble, PhD\\
  2013--18 & {\bf Graduate Research Assistant}, University of Maryland,
             Baltimore
             \newline Advisors: David Goodlett, PhD and Dudley Strickland, PhD\\

  2012--13 & {\bf Undergraduate Research Assistant}, University of North Carolina at
             Chapel Hill
             \newline Advisor: John Papanikolas, PhD\\
\end{tabular}

\mysection{Awards and Honors}
\begin{tabular}{\tdim}
  2021     & {\bf Rising Stars in Proteomics and Metabolomics: 40 under 40},
             \newline \textit{Journal of Proteome Research} \\
  2020     & {\bf Emerging Talent: Academia}, American Society for Mass Spectrometry\\ 
  2019--   & {\bf Ruth L Kirschstein Institutional National Research Service Award}
             \newline NIH T32HG000035, Postdoctoral Trainee\\
  2017     & {\bf Travel Fellowship to the May Institute on Computation and Statistics
             for\newline Mass Spectrometry and Proteomics}\\
  2017--18 & {\bf Ruth L Kirschstein Individual National Research Service Award}
             \newline NIH F31CA213815, PI\\
             % \newline \textit{Impact Score: 25 (16\%)}\\
  2016--17 & {\bf Ruth L Kirschstein Institutional National Research Service Award}
             \newline NIH T32HL007698, Predoctoral Trainee\\
  2012     & {\bf Markham Summer Undergraduate Research Award}\\
  2009     & {\bf Central Carolina's chapter of Phi Beta Kappa Scholarship}\\
\end{tabular}

\newpage

\mysection{Preprints}

\begin{etaremune}
  \item \textbf{Fondrie WE}, Noble WS. (2020) mokapot: Fast and flexible
    semi-supervised learning for peptide detection. \textit{biorRxiv}. DOI:
    10.1101/2020.12.01.407270.

\end{etaremune}

\mysection{Peer-Reviewed Publications}
{\small (* indicates equal contributions)}

\begin{etaremune}

  \item \textbf{Fondrie WE}, Noble WS. (2020) Machine learning strategy that
    leverages large datasets to boost statistical power in small-scale
    experiments. \textit{J Proteome Res} 6;19(3):1267--1274. PMID: 32009418.
  
  \item Liang T, Leung LM, Opene B, \textbf{Fondrie WE}, Lee YI, Chandler CE,
    Yoon SH, Doi Y, Ernst RK, Goodlett DR. (2019) Rapid microbial identification
    and antibiotic resistance detection by mass spectrometric analysis of
    membrane lipids. \textit{Anal Chem} 91(2):1286--1294. PMID: 30571097.

  \item \textbf{Fondrie WE}, Liang T, Oyler BL, Leung LM, Ernst RK, Strickland
    DK, Goodlett DR. (2018) Pathogen Identification Direct From Polymicrobial
    Specimens Using Membrane Glycolipids. \textit{Sci Rep} 8(1):15857. PMID:
    30367087.
    
  \item Liang T, Schneider T, Yoon SH, Oyler BL, Leung LM, \textbf{Fondrie WE},
    Yen G, Huang Y, Ernst RK, Nilsson E, Goodlett DR. (2018) Optimized surface
    acoustic wave nebulization facilitates bacterial phenotyping. \textit{Int J
      Mass Spectrom} 427:65--72.
    
  \item Au DT, Arai AL, \textbf{Fondrie WE}, Muratoglu SC, Strickland DK. (2018)
    Role of the LDL Receptor-Related Protein 1 in Regulating Protease Activity
    and Signaling Pathways in the Vasculature. \textit{Curr Drug Targets}
    19(11):1276--1288. PMID: 29749311.

  \item Au DT, Ying Z, Hernández-Ochoa EO, \textbf{Fondrie WE}, Hampton B,
    Migliorini M, Galisteo R, Schneider MF, Daugherty A, Rateri DL, Strickland
    DK, Muratoglu SC. (2018) LRP1 (Low-Density Lipoprotein Receptor-Related
    Protein 1) Regulates Smooth Muscle Contractility by Modulating Ca2+
    Signaling and Expression of Cytoskeleton-Related Proteins.
    \textit{Arterioscler Thromb Vasc Biol} 38(11):2651--2664. PMID: 30354243.
    
  \item Khan MM, Tran BQ, Jang Y, Park S, {\bf Fondrie WE}, Chowdhury K, Yoon
    SH, Goodlett DR, Chae S, Chae H, Seo S, Goo YA. (2017) Assessment of the
    therapeutic potential of persimmon leaf extract on prediabetic subjects.
    {\it Mol Cells} 40(7):466. PMID: 28743946.
    
  \item Leung LM, {\bf Fondrie WE}, Doi Y, Johnson JK, Strickland DK, Ernst RK,
    Goodlett DR. (2017) Identification of the ESKAPE pathogens by mass
    spectrometric analysis of microbial membrane glycolipids. {\it Sci Rep}
    7(1):6403. PMID: 28743946. 

  \item Clark DJ, {\bf Fondrie WE}, Liao Z, Yang AJ, Mao L. (2016) Triple SILAC
    quantitative proteomic analysis reveals differential abundance of cell
    signaling proteins between normal and lung cancer-derived exosomes. {\it J
      Proteomics} 133:161--169. PMID: 26739763.

  \item Clark DJ*, {\bf Fondrie WE*}, Liao Z, Hanson PI, Fulton A, Mao L, Yang
    AJ. (2015) Redefining the breast cancer exosome proteome by tandem mass tag
    quantitative proteomics and multivariate cluster analysis. \textit{Anal
      Chem} 87(20):10462--10469. PMID: 26378940. 
    
  \item Ma D, Bettis SE, Hanson K, Minakova M, Alibabaei L, {\bf Fondrie W},
    Ryan DM, Papoian GA, Meyer TJ, Waters ML, Papanikolas JM. (2013) Interfacial
    energy conversion in Ru(II) polypyridyl-derivatized oligoproline assemblies
    on TiO2. \textit{J Am Chem Soc} 135(14):5250--5253. PMID: 23514453. 
\end{etaremune}

\mysection{Additional Publications}
\begin{etaremune}
  \item {\bf Fondrie WE}. Biological Insight from Mass Spectrometry Through
    Novel Computational Approaches. Ph.D. dissertation. University of Maryland,
    Baltimore. June, 2018. Advisors: Dudley K. Strickland, Ph.D. and David R.
    Goodlett, Ph.D.
\end{etaremune}

\mysection{Patents}
\begin{etaremune}
  \item Goodlett DR, Ernst RK, Liang T, {\bf Fondrie WE}, Nilsson E. (2018)
    Methods for Lipid Extraction and Identification of Microbes Using Same Via
    Mass Spectrometry. US Patent Application 2017066342. Filed 12/14/2017. {\it
      Patent Pending.}
\end{etaremune}


\mysection{Scientific Software Contributions}
\begin{etaremune}
  
  \item {\bf mokapot}. A Python package that enables the fast and flexible
    application of semi-supervised learning to detect peptides in proteomics
    experiments. \\
    Homepage: \myref{https://github.com/wfondrie/mokapot} \\
    Roles: creator and maintainer

  \item {\bf ppx}. A Python package to retrieve metadata and files for
    proteomics projects from public repositories through ProteomeXchange. \\
    Homepage: \myref{https://github.com/wfondrie/ppx} \\
    Roles: creator and maintainer
    
 \item {\bf Crux}. A mass spectrometry toolkit for proteomics data. \\
   Homepage: \myref{http://crux.ms} \\
   Roles: contributor

 \item {\bf Percolator}. Semi-supervised learning for peptide identification
   from shotgun proteomics datasets. \\
   Homepage: \myref{http://percolator.ms} \\
   Roles: contributor

 \item {\bf dask-ml}. A library for distributed and parallel machine learning
   using dask. \\
   Homepage: \myref{https://ml.dask.org} \\
   Roles: contributor
   
 \item {\bf Kojak}. A search engine for cross-linking proteomics. \\
   Homepage: \myref{http://www.kojak-ms.org} \\
   Roles: contributor

 \item {\bf MSToolkit}. A library for interacting with proteomics data formats. \\
   Homepage: \myref{https://github.com/mhoopmann/mstoolkit} \\
   Roles: contributor

 \item {\bf ANN-SoLo}. A spectral library search engine for fast and accurate
   open modification searching. \\
   Homepage: \myref{https://github.com/bittremieux/ANN-SoLo} \\
   Roles: contributor
   
 \item {\bf spectrum\_utils}. A Python package for efficient MS/MS spectrum
   processing and visualization. \\
   Homepage: \myref{https://github.com/bittremieux/spectrum_utils} \\
   Roles: contributor.

 \item {\bf xComb}. A cross-linked peptide database approach to protein-protein
   interaction analysis. \\
   Homepage: \myref{https://github.com/GoodlettLab/xComb} \\
   Roles: maintainer and contributor

\end{etaremune}

\mysection{Talks}
\begin{etaremune}
  \item {\bf Fondrie WE}. Deep Learning to Enable Proteomics at Scale. American
    Society for Mass Spectrometry Emerging Talent: Academia webinar series,
    November 10--12, 2020. Online.
  
  \item {\bf Fondrie WE}. Learning from Mass Spectra. Genome Sciences
    Seminar, August 12, 2020. University of Washington Department of Genome
    Sciences, Seattle, Washington.
  
  \item {\bf Fondrie WE}, Leung LM, Strickland DK, Ernst RK, Goodlett DR.
    Detecting antibiotic resistance by MALDI-TOF analysis of bacterial membrane
    glycolipids. 65th American Society for Mass Spectrometry Annual Conference
    on Mass Spectrometry and Allied Topics, June 4--8, 2017. Indianapolis, IN.
    
  \item {\bf Fondrie WE}, Muratoglu SC, Hampton B, Migliorini M, Galisteo R,
    Strickland DK. LRP1 modulates TGF-$\beta$ signaling in the descending
    thoracic aorta. Molecular Medicine Research Retreat, October 6, 2016.
    Baltimore, MD.
\end{etaremune}

\mysection{Posters}
\begin{etaremune}
  \item {\bf Fondrie, WE}, Noble WS. Mokapot: Fast and Flexible Semi-Supervised
    Learning for Peptide Detection. 28th Conference on Intelligent Systems For
    Molecular Biology, July 13--16, 2020. Montreal, QC, Canada. \textit{Moved
      online due to COVID-19.}
  
  \item {\bf Fondrie, WE}, Noble WS. Boosting statistical power in small-scale
    experiments with Percolator. US Human Proteome Organization 16th Annual
    Conference. March 8--11, 2020. Seattle, WA \textit{Moved online due to
      COVID-19.}
  
  %\item {\bf Fondrie, WE}, Noble WS. Static Models Improve Percolator
  %  Performance on Small-Scale Proteomics Experiments. University of Washington
  %  Department of Genome Sciences Annual Retreat, Sept 18--20, 2019.
  %  Leavenworth, WA.
  
  \item {\bf Fondrie WE}, Noble WS. Robust Cross-Linked Peptide Detection Using
    Pretrained Neural Networks. 67th American Society for Mass Spectrometry
    Annual Conference on Mass Spectrometry and Allied Topics, June 1--6, 2019.
    Atlanta, GA.

  %\item {\bf Fondrie WE}, Zelter A, Henry E, Abel S, Le Roch KG, Davis TN, Noble
  %  WS. Building Robust Score Vectors for Cross-Linked Peptide Identification.
  %  University of Washington Department of Genome Sciences Annual Retreat, Sept
  %  17--19, 2018. Leavenworth, WA.

  \item {\bf Fondrie WE}, Hampton B, Muratoglu SC, Goodlett DR, Strickland DK.
    Detecting LRP1B protein interactions in glioma. UMB Cancer Biology Research
    Retreat, June 13, 2017. Baltimore, MD.

  \item {\bf Fondrie WE}, Muratoglu SC, Hampton B, Migliorini M, Galisteo R,
    Strickland DK. LRP1 modulates TGF-$\beta$ signaling in the descending
    thoracic aorta. GenTAC Thoracic Aortic Summit, September 22--23, 2016.
    Washington, DC.

  \item {\bf Fondrie WE}, Hampton B, Muratoglu SC, Goodlett DR, Strickland DK.
    Defining a mechanism of LRP1B tumor suppression in glioblastoma. UMB Cancer
    Biology Research Retreat, May 23, 2016. Baltimore, MD.

  \item {\bf Fondrie WE}, Clark DJ, Catania SM, Goo YA, Strickland DK, Goodlett
    DR. Investigating the regulated intramembrane proteolysis of LRP1B in
    glioblastoma progression through a proteogenomic approach. Mass Spectrometry
    in Biotechnology and Medicine, July 5--11, 2015. Dubrovnik, Croatia.

  \item {\bf Fondrie WE}, Clark DJ, Liao Z, Chen Y, Yang AJ. Novel
    identification of JAK1/STAT signaling proteins in breast cancer exosomes
    through shotgun proteomic analysis using multiple protein database search
    algorithms. UMB Cancer Biology Research Retreat, June 9, 2014. Baltimore,
    MD.
    
  \item {\bf Fondrie WE}, Bettis S, Ma D, Minakova M, Wilger D, Papoian G,
    Waters M, Papanikolas J. Flexibility matters: The role of scaffold tethers
    in Ru(II) and Os(II) chromophore separation. Southeastern Regional Meeting
    of the American Chemical Society, November 14--17, 2012. Raleigh, NC.
  \end{etaremune}


\mysection{Teaching}
\mysubsection{Training}
\begin{tabular}{\tdim}
  2020 & The Carpentries Instructor Training
         \newline \myref{https://carpentries.org/} \\
\end{tabular}

\mysubsection{Experience}
\begin{tabular}{\tdim}
  2021 & UW Winter Quarter Software Carpentry Workshop Instructor
         \newline \myref{https://uwescience.github.io/2021-01-11-uw-online/} \\

\end{tabular}
  
\newpage
\mysection{Mentoring}
\begin{tabular}{\tdim}
  \raggedright
  2021--     & {\bf Lincoln Harris}, PhD student\newline Department of Genome Sciences, University of Washington\\
  2020--     & {\bf Elena Romero}, PhD student\newline Department of Genome Sciences, University of Washington\\
  2020--     & {\bf Donovan See}, Undergraduate student\newline University of Washington\\
  2020--     & {\bf Lilian Heil}, PhD student\newline Department of Genome Sciences, University of Washington\\
  2019--     & {\bf Kianna Hales}, PhD student\newline Department of Genome Sciences, University of Washington\\
  
\end{tabular}

\mysection{Professional Service}

\mysubsection{Editorial Boards}
\begin{tabular}{\tdim}
  2021--23 & \textit{Journal of Proteome Research}. Editorial Advisory Board. \\
\end{tabular}

\mysubsection{Conference Organization}
\begin{tabular}{\tdim}
  2020 & Workshop on High Performance Computing, Big Data Analytics and
         Integration for Multi-Omics Biomedical Data (HPC-BOD 2020) \textit{Program
         Committee Member}.
\end{tabular}

\mysubsection{Professional Memberships}
\begin{tabular}{\tdim}
  2020--     & United States Human Proteome Organization (US HUPO) \\
  2013--     & American Society for Mass Spectrometry (ASMS) \\
\end{tabular}

\mysubsection{Departmental Service}
\begin{tabular}{\tdim}
  2019--     & UW Genome Sciences Art Committee \\
  2018--     & Postdoctoral representative in UW Genome Sciences\\ 
  2015--17   & UMB Molecular Medicine Event Planning Committee \\
  2017       & UMB Cancer Biology Research Retreat Organizing Committee \\
  2016       & UMB Grollman Lecture Organizing Committee \\
\end{tabular}

{\bf Journal Referee: }{Scientific Reports, Journal of Proteome Research}\\
{\bf Conference Referee: }{Intelligent Systems for Molecular Biology 2020,
  Research in Computational Molecular Biology 2021}

\newpage
\mysection{Scientific Blog Posts}
\begin{etaremune}
  \item {\bf Fondrie WE}. (2019) Using Static Models in Percolator. \\
    \myref{https://willfondrie.com/2019/11/using-static-models-in-percolator/}

  \item {\bf Fondrie WE}. (2019) An Intuitive Look at the xcorr Score
    Function in Proteomics. \\
    \myref{https://willfondrie.com/2019/02/an-intuitive-look-at-the-xcorr-score-function-in-proteomics/}

\end{etaremune}

\mysection{References}
{\bf William Noble} \\
Department of Genome Sciences, University of Washington\\
wnoble@uw.edu

{\bf Michael MacCoss} \\
Department of Genome Sciences, University of Washington\\
maccoss@uw.edu

%\newpage

{\bf Dudley Strickland} \\
Center for Vascular and Inflammatory Diseases, University of Maryland
Baltimore\\
dstrickland@som.umaryland.edu

{\bf David Goodlett} \\
Genome BC Proteome Centre, University of Victoria\\
goodlett@uvic.ca

\end{document}
  